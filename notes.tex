%%% Notes on this and that

\documentclass{article}

\usepackage{notes}

\addbibresource{mishmash.bib}

\title{Notes on this and that}

\hypersetup{pdftitle={Notes on this and that}}

\hypersetup{pdfsubject={This and that}}

\hypersetup{pdfkeywords={books, films, Himalaya, writing}}

\author{Raghavendra Nyshadham\thanks{\cczero} \\
  {\normalsize\nolinkurl{rn@raghnysh.com}}}

\hypersetup{pdfauthor={Raghavendra Nyshadham (rn@raghnysh.com)}}

\date{2024-09-13}

\begin{document}

\begin{titlingpage}
  \maketitle

  \begin{abstract}
    This document contains my notes on this and that, such as the
    books that I read, and the films that I watch.  I also hope to
    write about the Himalaya: the Garhwal Himalaya, some of which I
    have seen, and near whose foothills I live.  The document is a
    work in progress.  The files of the current version of the
    document are available at
    \url{https://github.com/raghnysh/mishmash}.

    \ghtopics{books, films, Himalaya, writing}
  \end{abstract}
\end{titlingpage}

\tableofcontents

\section{Introduction}
\label{sec:113nrd0o}

This document contains my notes on this and that, such as the books
that I read, and the films that I watch.  I also hope to write about
the Himalaya: the Garhwal Himalaya, some of which I have seen, and
near whose foothills I live.  The document is a work in progress.  The
files of the current version of the document are available at
\url{https://github.com/raghnysh/mishmash}.

\section{Mountains may depart}
\label{sec:ov8yna0s}

I first watched this film on 26 July, 2024 on MUBI.  It is directed by
Jia Zhangke, and was made in 2015.  The main actors in the film are
Zhao Tao, Yi Zhang, Jingdong Liang, and Dong Zijian.  Tao had acted in
some of Zhangke's earlier films including \emph{A touch of sin}.

The central story of \emph{Mountains may depart} invloves a woman Shao
Tao (played by Zhao Tao), the men Jinsheng Zhang and Jangjung Liang
a.k.a Liangzi (played by Yi Zhang and Jingdong Liang), and the boy
Daole Zhang a.k.a Dollar (played by Dong Zijian).  The film is located
mostly in Fenying, where Zhangke grew up; it is a city in the Shanxi
province of North China.

\subsection{Youth}
\label{sec:jfdg3q3z}

It is 1999.  Tao is a young woman, a singer and a dancer.  She teaches
in a disco school in the city, and sings regularly at the New Year
concert.  Zhang and Liangzi are her close friends, both of them in
love with her, although she seems strangely oblivious to their
feelings.

The film opens with Tao and her students dancing to the famous Pet
Shop Boys song \emph{Go West}.  She makes waves with her hands as the
dancers move with exercise-like regularity and disappear from the
scene holding each other in single file.

Zhang is an aggressive industrial entrepreneur, and goes around
bragging about the money he makes and his fancy foreign car.  Liangzi
works as a clerk in a coal mine, keeping track of the helmets of the
miners.

Neither of the two men is reticent about his love for Tao.  She
doesn't seem to notice these emotions.  Perhaps she realises that
acknowledging their feelings would force her to choose one of them and
alienate the other.  She tries her best to steer the middle course.
So we see the three of them hanging out as Tao prepares to sing at the
New Year festival.  Zhang shows them his new car, and later takes them
for a spin in it.  They burst fireworks on the frozen Fen river.

The tense bonhomie between the three is not destined to last.  Already
on their Fen river outing, Zhang tells Tao that he would like to be
alone with her, and that he does not like their triangular tie with
Liangzi.  Tao retorts that the triangle is a stable arrangement, and
is averse to distancing herself from Liangzi.  As the fireworks light
up the sky over the white river, we see the three friends standing
away from each other, looking up from the vertices of a triangle.

When the mine that employs Liangzi goes bankrupt, Zhang buys it and
fires him.  Tao soon finds herself forced by Zhang to make a choice.
She visits Liangzi's home with an invitation to her wedding with
Zhang.  Liangzi sees that there is nothing for him in Fenyang, and
leaves the city without bidding Tao goodbye.

Zhang and Tao soon have a child, a boy they name Daole, with the
nickname Dollar in honour of Zhang's capitalist ambitions.  They also
adopt a dog, a white Labrador who Zhang says has a lifespan of 15
years.

\subsection{The middle years}
\label{sec:m1ubz2fo}

It is 2014.  Tao's Labrador, though 15 and old, is still with her.
Zhang and Dollar aren't.  Tao and Zhang got divorced, and the father
won custody of the child.  Tao continues to live in Fenyang, running a
petrol station.  Zhang has moved to Shanghai, where Dollar attends the
International School.

Liangzi has had a difficult life, a migrant labourer moving from one
back-breaking and ill-paying job to another.  He is working in a coal
mine again now, and is married with a small child.  He has acquired a
rasping cough that never seems to leave him.  It is cancer of the
lungs, but he and his devoted wife Xiao Lu (played by Liu Lu) don't
have the money for chemotherapy and medicines.

Soon Liangzi is unable to work, and the family return to his home in
Fenyang.  Unable to see his suffering, and guessing at her husband's
first love, Lu approaches Tao and asks her to help.  Tao visits them
and hands over the money for the treatment.

Liangzi disappears from the story at this juncture.  It is surprising
to see a central character going away when about half the film is yet
to go.  I assume he got better with the chemotherapy and the
medicines.  Or perhaps he died, loyal lover that he was, broken by the
harrowing and lonely years on the road.

Age is catching up with Tao.  Her father dies while visiting an old
friend in another city.  Tao travels there to bring his body back to
be cremated in Fenyang.  She speaks formally with the doctor at the
mortuary, but breaks into rending tears in front of her father's
friend.  I wonder why he didn't hold her, comfort her.  Alone, she
brings her dead father back in an ambulance.

Tao's young son visits Fenyang from Shanghai for his grandfather's
funeral.  He looks just about six, but Tao is impatient with him.  She
calls his habitual scarf girlish, and yells at him in public when he
fumbles with the funeral rituals.  She stops his phone calls to his
stepmother in Shanghai.  One feels sorry for the child, but they
realise their love for each other as she takes the train to Shanghai
with him.  He rests his head on her shoulder, and, later, sleeping on
the berth, plays with the keys to her house that she has presented
him.

\subsection{Old age}
\label{sec:7wgsj9bn}

It is 2025.  Tao is old.  Her Labrador is gone, but he seems to have
left one of his descendants, a white puppy, to look after her.

Zhang has moved his family to Melbourne (although the scenes of this
part were filmed more than 2000 miles away in Bunbury), and his second
wife also seems to have separated from him.  He has grown old, and
spends his time drinking with his Chinese-Australian friends in his
designer-home.  He doesn't know much English, and talks to his son in
Mandarin.

Dollar is now a teenager.  He goes to a posh school, doesn't speak his
mother tongue, and barely understands his father's words.  Their
relationship has turned fractious, the differences between them
exacerbated by the difficulties of Dollar's adolescence in a foreign
environment.  He is so traumatised by Tao's absence that he cannot
bear to mention her, and says that he is a test-tube baby without a
mother.

At school, Dollar has an easy equation with his friends, several of
them of Chinese origin.  Mia (played by Sylvia Chang) teaches these
children Mandarin and Chinese culture.  The kids are bright, and there
is a playful repartee between them and her in the class.

Dollar works the evenings at a Chinese restaurant, and sometimes
delivers dinner to Mia's apartment.  Over time they become close, and
fall in love with each other.  The relationship between the
middle-aged woman and the teen is at times disturbing to watch, but
Mia is good to Dollar.  She travels with him, and tells him that he
should go to Fenyang to meet Tao.  He is hesitant because of the long
separation from his mother.

The affair between Dollar and Mia soon succumbs to the difference
between their ages.  As they part from each other in Melbourne, Dollar
at last utters his mother's name, ``Tao''.

Tao is living alone in Fenyang.  We see her mincing meat in her
kitchen as she has always done.  Sometimes she hears voices calling
out her name.  She smiles at her fancies, and takes her Lab out for a
walk on the frozen Fen.

The river is much like what it was that New Year in 1999 when she
described a triangle with Zhang and Liangzi.  The water has turned to
ice again.  Tao removes the leash from her dog.  She raises her head
to the sky, and dances slowly, as she did long ago in the disco, to
Pet Shop Boys singing \emph{Go West}.  That's how the film ends.

\subsection{The song}
\label{sec:njpsvlf6}

The song \emph{Go West} by Pet Shop Boys opens the film and closes it
as well.  It was originally written and sung in 1979 by an American
disco music group called Village People.  Pet Shop Boys extended the
text of that song and released the new version in 1993.  It has
attracted much attention since then, and has been identified with gay
culture and football fandom \parencite{bib:3hf0kuo0}.

The official video of the song \parencite{bib:4dsv5d40} shows a relief
sculpture of Lenin, goose-stepping soldiers, and several other images
from the iconography of the Soviet Union.  Robot-like humans march
across a flight of stairs towards a `W' sign while a woman dressed
like the Statue of Liberty vocalises.  It sounds like a sarcastic take
on the fall of the USSR and the false promise that the capitalist West
held then for the people of the communist East.

For me, the song embodies hope, the possibility of a better time and
place than now.  Some of the lyrics that were written by Pet Shop Boys
for the song go
\begin{verse} \itshape
  There where the air is free \\
  we'll be what we want to be \\
  Now if we make a stand \\
  we'll find our promised land
\end{verse}
There is a reassurance here that collective action will give us the
happiness that we seek individually.  One of the best renditions of
the song that I have heard is by the Brythoniaid Male Voice Choir from
Wales \parencite{bib:4lizl8c7}.  The sight of old men from the choir
singing these lines \parencite{bib:n7kfye4j} suggests wistful memories
of a dream for the future.

Zhangke talks about `Go West' in an interview about `Mountains may
depart' \parencite{bib:zt0dmft0}.  He speaks of how discos became
common in Beijing when he was a student there in the late 90s.  The
song was played at midnight, and the dancers formed a chain to the
music as in the opening scenes of the film.  He says that the song
evokes a sense of freedom that is possible only in youth: ``When
you're young and free, it doesn’t matter if you're going `west' or
elsewhere.  The important thing is the impulsiveness with which you
{`go.'}''

When I watched the closing scenes of the film, where Tao frees her dog
before dancing to the song in the snow, I thought that it was a sign
that she was going to die in the freezing cold.  Later, I learnt that
the director wants to convey something else altogether
\parencite{bib:6xz64ynw}.  Zhangke says, ``At the end of the script I
originally envisioned a day that was snowing and Tao is coming home
from getting groceries.  She hears someone call her and she turns
around and doesn’t see anybody.  Then I thought that perhaps she could
dance, because even though she’s ended up as a very lonely woman, she
still has to live on like all of us do.  In her heart there is still a
vitality and I wanted to draw that out.''

As for the most important thing in Tao's life, her son, did Dollar
overcome the pain of the childhood away from her, and travel to
Fenyang to meet her? We are not told, but I don't think he did.
Perhaps it doesn't matter anymore now, as she moves gently in the
falling snow.

\bibsection

\end{document}

%%% End of file
